\documentclass[10pt]{scrartcl}
\usepackage{psfrag,amsmath,amsfonts,verbatim,mathtools}
%\usepackage[margin=.85in]{geometry}
\usepackage{fullpage}
\usepackage[parfill]{parskip}
\usepackage{bookman}
\usepackage[small,bf]{caption}
\usepackage[shortlabels]{enumitem}
\usepackage{xfrac}
\usepackage{commath}
\usepackage{titlesec}
\usepackage{physics}
\usepackage{courier}
\usepackage[hang]{footmisc} 
\usepackage{caption}
\newenvironment{sysmatrix}[1]
 {\left[\begin{array}{@{}#1@{}}}
 {\end{array}\right]}
\newcommand{\ro}[1]{%
  \xrightarrow{\mathmakebox[\rowidth]{#1}}%
}
\newlength{\rowidth}
\AtBeginDocument{\setlength{\rowidth}{3em}}

\newcommand{\ones}{\mathbf 1}
\newcommand{\reals}{{\mbox{\bfseries R}}}
\newcommand{\integers}{{\mbox{\bf Z}}}
\newcommand{\symm}{{\mbox{\bf S}}} 
\newcommand{\Mod}[1]{\ (\mathrm{mod}\ #1)}

\newcommand{\nullspace}{{\mathcal N}}
\newcommand{\range}{{\mathcal R}}
\newcommand{\Rank}{\mathop{\bf Rank}}
\newcommand{\diag}{\mathop{\bf diag}}
\newcommand{\card}{\mathop{\bf card}}
\newcommand{\proj}{\mathop{\bf Proj}}
\newcommand{\conv}{\mathop{\bf conv}}
\newcommand{\zero}{\mathop{\bf 0}}
\newcommand{\prox}{\mathbf{prox}}

\newcommand{\Expect}{\mathop{\bf E{}}}
\newcommand{\Prob}{\mathop{\bf Prob}}
\newcommand{\Co}{{\mathop {\bf Co}}} 
\newcommand{\dist}{\mathop{\bf dist{}}}
\newcommand{\argmin}{\mathop{\rm argmin}}
\newcommand{\argmax}{\mathop{\rm argmax}}
\newcommand{\epi}{\mathop{\bf epi}} 
\newcommand{\Vol}{\mathop{\bf vol}}
\newcommand{\dom}{\mathop{\bf dom}} 
\newcommand{\intr}{\mathop{\bf int}}
\newcommand{\sign}{\mathop{\bf sign}}

\newcommand{\cf}{{\it cf.}}
\newcommand{\eg}{{\textit e.g.}}
\newcommand{\ie}{{\textit i.e.}}
\newcommand{\etc}{{\it etc.}}
\bibliographystyle{alpha}


\usepackage{amsmath}



\usepackage{titling}
\setlength{\droptitle}{-1cm}



\title{Homework 1 for Physics  531}
\author{Ashok M. Rao}

\begin{document}
\maketitle
\pagenumbering{gobble}

\paragraph{Problem 1}
\begin{enumerate}[(a)]
\item Expand the operator as $\bra{(AB)^{\dagger}a'}\ket{a''}=\mel{a'}{((AB)^{\dagger})^{\dagger}}{a''}$ then conclude
\begin{align*}
	\mel{a'}{AB}{a''}&= \braket{B^\dagger A^\dagger a'}{a''}
\end{align*}
So $(AB)^\dagger = B^\dagger A^\dagger$. 
\item Likewise
\begin{align*}
\Trace{AB} &=\sum_{a'}\mel{a'}{AB}{a'} \\
&= \sum_{a', a''}\mel{a'}{A}{a''}\mel{a''}{B}{a'}\\
&= 	\sum_{a', a''}\mel{a''}{B}{a'}\mel{a'}{A}{a''}\\
&=\sum_{a''}\mel{a''}{BA}{a''}
\end{align*}
which evaluates to $\Trace{BA}$.
\item Using either of the above
\begin{align*}
	\Trace{U^{\dagger} A U} &= \Trace{U U^\dagger A} \\
	&= \Trace{A}
\end{align*}
\item Express the operator function as $f(A) = \sum_{a'} f(a')\ket{a}\bra{a}$ using completeness.
\end{enumerate}

\paragraph{Problem 2}
\begin{enumerate}[(a)]
	\item By inspection it is clear that $\Trace{\sigma_i} = 0$ and $\det{\sigma_i}=-1$ for each Pauli matrix $\sigma_i$. Given that each is two-dimensional it is easy to conclude that all eigenvalues are like $\lambda_{ii}=\pm 1$. 
	\item Compute that 
\begin{align*}
	\mqty(\pmat{1}) \mqty(\pmat{2}) +\mqty(\pmat{2}) \mqty(\pmat{1}) &= 2\mqty(-i & 0\\0 & i) \\
	\mqty(\pmat{1}) \mqty(\pmat{3}) -\mqty(\pmat{3}) \mqty(\pmat{1}) &= -2\mqty(0 & 1\\-1 & 0) \\
	\mqty(\pmat{2}) \mqty(\pmat{3}) +\mqty(\pmat{3}) \mqty(\pmat{2}) &= 0
\end{align*}
Directly, $\sigma_i\sigma_j = \delta_{ij} + i\epsilon_{ijk}\sigma_k$ follows To find the commutator and anticommutator compute that
\begin{align*}
	\comm{\sigma_i}{\sigma_j} &= \delta_{ij} + i\epsilon_{ijk}\sigma_k - \delta_{ji} + i\epsilon_{ijk}\sigma_k \\
	&= 2i\epsilon_{ijk}\sigma_k \\
	\acomm{\sigma_i}{\sigma_j}&= \delta_{ij} + i\epsilon_{ijk}\sigma_k + \delta_{ji} - i\epsilon_{ijk}\\
	&= 2I\delta_{ij}
\end{align*}
\item Expand and calculate
\begin{align*}
	\exp{i\vb{a}\cdot\va{\sigma}} &= \sum_{n=0}^{\infty}\frac{(-1)^n(\vb{a}\cdot\va{\sigma})^{2n}}{(2n)!} + i\sum_{n=0}^{\infty}\frac{(-1)^n(\vb{a}\cdot\va{\sigma})^{2n}\cdot(\vb{a}\cdot\va{\sigma})}{(2n+1)!} \\
	&= \sum_{n=0}^{\infty}\frac{(-1)^n\abs{\vb{a}}^{2n}}{(2n)!} + i\sum_{n=0}^{\infty}\frac{(-1)^n\abs{\vb{a}}^{2n+1}}{(2n+1)!}\left(\frac{\vb{a}\cdot\va{\sigma}}{\abs{\vb{a}}}\right)\\
	&= \cos{\abs{\vb{a}}}+i\frac{\vb{a}\cdot\va{\sigma}}{\abs{\vb{a}}}\sin{\abs{\vb{a}}}
\end{align*}
wherein (b) is used to simplify $(\vb{a}\cdot\va{\sigma})^{2n}$ since the cross term in $(\vb{a}\cdot\va{\sigma})^{2} = \sum_{i, j}\vb{a}_i \vb{a}_j\sigma_i\sigma_j = \sum_{i,j} \delta_{ij}\vb{a}_i \vb{a}_j$ due to the fact that $\sigma_i\sigma_j + \sigma_j\sigma_i = 0$ 
\end{enumerate}

\paragraph{Problem 3}
\begin{enumerate}[(a)]
\item Simply decomposing $H$ (without assuming anything further), note that
\begin{align*}
	H = \frac{1}{2}\left[(h_{11} + h_{22})I + (h_{12}+h_{21})\sigma_x + i(h_{12}-h_{21})\sigma_y + (h_{11}-h_{22})\sigma_z\right]
\end{align*}
Since we have not assumed anything about $H$, the coefficients may be complex. Noting that the trace of $\sigma_i=0$ and for convenience letting $\sigma_0=I$,
\begin{align*}
	a_k &= \frac{1}{2}\Trace{(\sigma_k H)}
\end{align*}
Yet comparing this with the decomposition above, it's clear that Hermiticity of $H$ is one and the same as requiring real $a_0$ and $\vb{a}$.
\item Without loss of generality express $H$ as 
\[H = \mqty(a_0 + a_3 & a_1 - i a_2 \\ a_1 + i a_2 & a_0 - a_3)\]
Solving the characteristic polynomial,
\begin{align*}
	\det{H-\lambda I} &=\lambda^2 -2a_0\lambda + \left[(a_0+a_3)(a_0-a_3)-a_1^2 -a_2^2\right]\\
	&= \lambda^2 - \lambda\Trace{H} + \det{H}
\end{align*}
Solving the quadratic,
\begin{align*}
	\lambda &= \frac{1}{2}\left(\Trace{H}\pm\sqrt{\Trace{H}^2 - 4\det{H}}\right)\\
	&= a_0\pm\sqrt{\vb{a}\cdot\vb{a}}
\end{align*}
\item In the additive form $H=a_0 I + \vb{a}\cdot\va{\sigma}$, it is clear that the eigenvectors of $H$ are the same as those for $\vb{a}\cdot\va{\sigma}$ since the first term is just an additive shift. The magnitude on $\sqrt{\vb{a}\cdot\vb{a}}$ may be similarly disregarded for convenience so that the norm goes to unity. We solve for
\[\ket{H;\pm}=\pm\mqty(a_3 & a_1 - i a_2\\a_1 + i a_2& a_3)\ket{H;\pm}\]
Solve the problem and this may be written as 
\begin{align*}
\ket{H;\pm}&=\qty(\pm\sqrt{a_1^2 +a_2^2}, {a_1 + ia_2}) \\
&=	\qty(\pm\sqrt{(1+a_3)(1-a_3)}, {a_1 + ia_2}) \\
&= \frac{1}{\sqrt{2}}\qty(\sqrt{1\pm a_3}, \pm\sqrt{1\mp a_3}\cdot(a_1 + i a_2))\\
&= \frac{1}{\sqrt{2}}\qty(\sqrt{1\pm a_3}, \pm\sqrt{1\mp a_3}\exp{i\alpha})
\end{align*}
Further, from the eigenvalue condition we have that $a_3=\pm (1 + \sqrt{a_1^2 + a_2^2})$ or equivalently $a_1^2 + a_2^2 = (1-a_3^2)\cos^2{\beta} + (1-a_3^2)\sin^2{\beta}$ giving 
\[\ket{H;+}=\mqty(\cos{(\beta/2)}\\ \sin{(\beta/2)}\cdot\exp{i\alpha}),\quad \ket{H;-}=\mqty(\sin{(\beta/2)}\\-\cos{(\beta/2)}\cdot\exp{i\alpha})\]
Clearly these are orthogonal and stated both in terms of $a_k$ as well as angles to the axes.
\item This was done above simply to handle automatically the singularity case where $1+a_3=0$ though as $\beta=\theta$ and $\alpha =\phi$. 
The eigenvalues are as before though by a shift,
\[\lambda = a_0 \pm \sqrt{A^2(\vu{n}\cdot\vu{n})} = a_0 \pm A\]
\end{enumerate}

\paragraph{Problem 4}
\begin{enumerate}[(a)]
	\item The set of eigenvalues is $\{-1, 0, 1\}$ which are the only observable measurements.
	\item Calculate that
	\begin{align*}
		\expval{L_x}&=\mel{L_z=1}{L_x}{L_z=1} = 0\\
		\expval{L_x^2}&=\frac{1}{2}\mqty(1\\0\\0)^T\mqty(0&1&0\\1&0&1\\0&1&0)^2\mqty(1\\0\\0)\\
		&= \frac{1}{2}\mqty(1\\0\\0)^T\mqty(1&1&0\\1&1&0\\1&0&01)\mqty(1\\0\\0)\\
		&=\frac{1}{2}
	\end{align*}
This gives a calculation for the variance
\[\expval{(\Delta L_x)^2} = \expval{L_x^2}-\expval{L_x}^2 = \frac{1}{2}\]
so that the standard deviation is $\Delta L_x = \frac{1}{\sqrt{2}}$.
\item Since $L_x$ and $L_z$ are compatible operators, they share eigenvalues. The corresponding eigenvectors, in the order of $\{-1, 0, 1\}$ can be found by inspection, and then normalizing gives:
\begin{align*}
	\lambda_1 = \qty(1/2, -1/\sqrt{2}, 1/2),\quad \lambda_2 = \qty(1/\sqrt{2},0,-1/\sqrt{2}),\quad \lambda_3=\qty(1/2, 1/\sqrt{2}, 1/2)
\end{align*}
	\item The probability of each outcome would be given by $\abs{\braket{L_x}{L_z}}^2$. By symmetry we can conclude that
	\[\abs{\braket{L_x=-}{L_z=-}}^2 = \abs{\braket{L_x=+}{L_z=-}} = \left(\frac{1}{2}\right)^2\]
Thus measuring the positive and negative state both have probability $(1/4)$ and the remaining density falls on measuring $L_x=0$ which has probability $(1/2)$. 
\item $L_x^2$ is diagonal and can be multiplied by inspection, and also it is straightforward that it has eigenvalues $\{0, 1\}$. Thus it must be that the state after the state is $\ket{L_z=\pm 1}$.  Calculate the operator as
\item \[\left(\ket{L_z=1}\bra{L_z=1} + \ket{L_z=-1}\bra{L_z=-1}\right)\cdot\ket{\psi}\]
Using the solution from (b) calculate the result to be as $\qty(1/2,0,1/\sqrt{2})$ or which normalizes to $\ket{\psi}=\qty(2/2\sqrt{3},0,2/\sqrt{6})$.  Thus the probability of measuring the two possibilities $L_z=1$ and $L_z=-1$ is $1/3$ and $2/3$ respectively.
\item Let $\abs{\braket{L_z}{\psi}}$ be coefficients based on the stipulated probabilities. In particular, $\psi = \sum_{a'} a'\ket{L_z}$ so that we can write the coefficients respectively as $\abs{a_{-}}^2 = \abs{a_{+}}^2 = 1/4$ (as from above) and likewise $\abs{a_0}^2 = 1/2$. Taking square roots as in $\sqrt{\abs{a_{+,-,-}}^2}$ yields the coefficients for the expansion on the problem set as required. Finally calculate that
\item Resubstitute from (b) into (g) so that 
\begin{align*}
 \abs{\braket{L_x=0}{\psi}}^2 &= \frac{1}{2}\left(\mqty(1\\0\\-1)^T\mqty((1/2)\exp{i\delta_1}\\(1/\sqrt{2})\exp{i\delta_2}\\ (1/2)\exp{i\delta_3})\right)^2	\\
 &=\frac{1}{8}\abs{\exp{i\delta_i}-\exp{i\delta_3}}2
 \end{align*}
 Apply the identity that $\abs{e^{ix}-e^{iy}}^2 = \left[\Im{e^{ix}}-\Im{e^{iy}}\right]^2 + \left[\Re{e^{ix}}-\Re{e^{iy}}\right]^2$ to conclude:
 \[P(L_x=0) = \frac{1}{4}\left(1 + \Re{\exp{i(\delta_1-\delta_3)}}\right)=\frac{1+\cos(\delta_1-\delta_3)}{4}\]
 where the middle result follows from applying the Pythagorean identity on the squared terms, and double-angle identities on the cross terms. In fact, $\delta_2$ is irrelevant though the phase difference between $\delta_1$ and $\delta_3$ affects the final outcome. In particular, the probability $P(L_x=0) = 1/2$ is the maximum and attained when $\delta_1=\delta_3$. Contrarily when $\delta_1 = \delta_3 + \frac{\pi}{2}$ the probability is minimized to $P(L_x=0)=0$, thus the result is entirely determined by relative not absolute phase.
\end{enumerate}

\end{document}